\chapter{Materials and Methodology}
\section{System Design}
\paragraph{} The system deals with fetching, summarization and creation of front end results for the user. We have used extensive use of python libraries to aid us in parsing data from RSS feeds and extracting data from websites. 
\paragraph{} Next stage was to cluster the documents fetched based on the topics. So we're able to generate input documents for the summarizer. The summarizer using the below mentioned algorithms and methods generate a summery based on length specified. Finally a JSON file is generated for the generated summaries. 
\paragraph{} An HTML front end using AJAX is used to fetch this JSON file to the user. The block diagram given below will make the whole procedure clearer.
\section{Platform Details}
\subsection{Python}

\paragraph{} Python is a general purpose, high-level programming language whose design philosophy emphasizes code readability. Python's syntax allows programmers to express concepts in fewer lines of code than would be possible in languages such as C and the language provides constructs intended to enable clear programs on both a small and large scale. Python is a programming language that lets you work more quickly and integrate your systems more effectively. Python has small and has clean source codes. Standard library is full of useful modules.
\paragraph{} Python supports multiple programming paradigms, including object-oriented, imperative and functional programming styles. It features a dynamic type system and automatic memory management and has a large and comprehensive standard library.
\paragraph{} Like other dynamic languages, Python is often used as a scripting language, but is also used in a wide range of non-scripting contexts. Using third-party tools, Python code can be packaged into standalone executable programs. Python interpreters are available for many operating systems.
\textbf{Syntax and Semantics}
\paragraph{} The syntax of the Python programming language is the set of rules that defines how a Python program will be written and interpreted (by both the runtime system and by human readers). Python was designed to be a highly readable language. It has a relatively uncluttered visual layout and uses English keywords frequently where other languages use punctuation. Python aims towards simplicity and generality in the design of its syntax.
\paragraph{} Python uses whitespace indentation, rather than curly braces or keywords, to delimit blocks; a feature also termed the off-side rule. An increase in indentation comes after certain statements; a decrease in indentation signifies the end of the current block.
\textbf{Features of Python}
\begin{enumerate}[1. ]
\item \textbf {Simple} \newline Python is a simple and minimalistic language. Reading a good Python program feels like reading English, although very strict English! This pseudo-code nature of Python is one of its greatest strengths.

\item \textbf{Easy to Learn} \newline As we will discover in this book, Python is very easy to get started with and has an extraordinarily simple syntax.
\item \textbf{Free and Open Source} \newline Python is an example of an open source software. In simple terms, you can freely distribute copies of the software, read the source code, make changes to it and use pieces of it in new free programs. Open source is based on the concept of a community which shares knowledge. This is one of the reasons why Python is so good - it is constantly improved by a community which just wants to see a better Python.
\item \textbf{High-level Language} \newline When you write programs in Python, you do not need to bother about low-level details such as managing the memory used by your program, etc.
\item \textbf{Portable} \newline Due to its open source nature, Python has been ported to (i.e. changed to make it work on) many platforms. All your Python programs can work on any of these platforms without requiring any changes as long as you are careful to avoid any system-specific features.
\item \textbf{Interpreted} \newline Python, on the other hand, does not need the compilation and linking/loading steps. You just run the program directly from source code. Internally, Python converts the source program into an intermediate form called bytecodes and then translates this into the native language of your specific system and then runs it. All this actually makes Python much easier to use since you do not have to worry about compiling the program, making sure the proper libraries are linked and loaded, etc. This also makes your Python programs more portable since you can just copy the program to another computer and it just works!
\item \textbf{Object Oriented} \newline Python supports both procedure-oriented programming as well as object-oriented programming. In procedure-oriented programming, the program is built around procedures or functions which are just reusable pieces of programs to which data is fed. In object-oriented programming, the program is built around objects which combine both data and functionality.


\item \textbf{Extensible} \newline If you need a critical piece of code in your program to run very fast or want to have a piece of algorithm to be hidden from the outside world, then you can write that part of the program in languages like C or C++ and then use that part from your Python programs.
\item \textbf{Embeddable} \newline You can embed Python in your programs written in other languages like C or C++ to give 'scripting' capabilities for your program's users.
\item \textbf{Extensive Libraries}
\end{enumerate}
\subsection{NLTK}
\paragraph{} The Python Standard Library is huge. It can help you with regular expressions, documentation generation, unit testing, threading, databases, web browsers, CGI, FTP, email, XML, HTML, WAV files, cryptography, GUI using Tk, and  many other system-specific functionality as well.
\paragraph{} NLTK is a leading platform for building Python programs to work with human language data. It provides easy-to-use interfaces to over 50 corpora and lexical resources such as WordNet, along with a suite of text processing libraries for classification, tokenization, stemming, tagging, parsing, and semantic reasoning.
\paragraph{} Thanks to a hands-on guide introducing programming fundamentals alongside topics in computational linguistics, NLTK is suitable for linguists, engineers, students, educators, researchers, and industry users alike. NLTK is available for Windows, Mac OS X, and Linux. Best of all, NLTK is a free, open source, community-driven project. 
\paragraph{} NLTK has been called “a wonderful tool for teaching, and working in, computational linguistics using Python,” and “an amazing library to play with natural language.”
\paragraph{} Natural Language Processing with Python provides a practical introduction to programming for language processing. Written by the creators of NLTK, it guides the reader through the fundamentals of writing Python programs, working with corpora, categorizing text, analyzing linguistic structure, and more.

\begin{enumerate}[1. ]
\item \textbf{NLTK Stemmers} \newline Interfaces used to remove morphological affixes from words, leaving only the word stem. Stemming algorithms aim to remove those affixes required for eg. grammatical role, tense, derivational morphology leaving only the stem of the word. This is a difficult problem due to irregular words (eg. common verbs in English), complicated morphological rules, and part-of-speech and sense ambiguities (eg. ceil- is not the stem of ceiling).
\item \textbf{NLTK Tokenizer Package} \newline Tokenizers divide strings into lists of substrings. For example, tokenizers can be used to find the list of sentences or words in a string. NLTK tokenizers can produce token-spans, represented as tuples of integers having the same semantics as string slices, to support efficient comparison of tokenizers. (These methods are implemented as generators.)
\item \textbf{NLTK Taggers} \newline This package contains classes and interfaces for part-of-speech tagging, or simply “tagging”. A “tag” is a case-sensitive string that specifies some property of a token, such as its part of speech. Tagged tokens are encoded as tuples (tag, token). This package defines several taggers, which take a token list (typically a sentence), assign a tag to each token, and return the resulting list of tagged tokens. Most of the taggers are built automatically based on a training corpus
\item \textbf{Corpus} \newline A large corpus can provide a wide variety of useful information, provided that there are decent tools to extract it. In Natural Language Processing (NLP), for example, statistical information obtained from large corpora (consisting of tens of millions of words) is used to inform many different tasks, ranging from guessing the most likely parsing for a sentence to determining the likelihood that a document matches key terms in a search.
\end{enumerate}
\paragraph{} NLTK has been used successfully as a teaching tool, as an individual study tool, and as a platform for prototyping and building research systems.
\subsection{SciPy-Cluster}
\paragraph{}This library provides Python functions for agglomerative clustering. Its features include  
generating hierarchical clusters from distance matrices 
computing distance matrices from observation vectors 
computing statistics on clusters 
cutting linkages to generate flat clusters 
\subsection{NumPy}
\paragraph{} NumPy is an extension to the Python programming language, adding support for large, multi-dimensional arrays and matrices, along with a large library of high-level mathematical functions to operate on these arrays. The ancestor of NumPy, Numeric, was originally created by Jim Hugunin with contributions from several other developers. In 2005, Travis Oliphant created NumPy by incorporating features of Numarray into Numeric with extensive modifications. NumPy is open source and has many contributors.



    