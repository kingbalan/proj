\section{Literature Review} 
\paragraph{} With rapid growth of internet and the upcoming of internet services since 1995 ,the amount of data and information being shared and accessed  throughout the world has increased rapidly. An efficient method to improve the viewablity and accessibility of documents was required and Multi-document Summarization emerged as one the main tool. Research in this field was started back then and none of the designed algorithms could satisfy the entire requirements, each had shortcoming in certain fields. Hence a proper combination of various Algorithms was required to obtain a decent result and we have implemented ours in that way. 
 
\paragraph{} A straightforward approach in Multi-document Summarization is to first cluster the documents and then summarize each document cluster using summarization methods. However, most of the current summarization methods are solely based on the sentence-term matrix and ignore the context dependence of the sentences. As a result, the generated summaries lack guidance from the document clusters. Dingding Wang    proposed a new language model to simultaneously cluster and summarize documents by making use of both the document-term and sentence- term matrices. By utilizing the mutual influence of document clustering and summarization, his method makes a better document clustering method with more meaningful interpretation and an effective document summarization method with guidance from document clustering .  
 
\paragraph{} Traditional clustering techniques such as hierarchical and partitioning methods have been used in clustering documents. Hierarchical clustering proceeds successively by building a tree of clusters using bottom-up or top-down approaches. For example, hierarchical agglomerative clustering (HAC) \cite{Duda} is a typical bottom-up hierarchical clustering method, which takes each document as a singleton cluster to start off with and then merges pairs of clusters until all clusters have been encapsulated into one final cluster that contains all documents. Partitioning methods attempt to directly decompose the docu-ment collection into a number of disjoint classes such that the documents in a cluster are more similar to one another than the documents in other clusters \cite{HeEtAl}. For example, K-means \cite{Duda} is a typical partitioning method, which aims to minimize the sum of the squared distances between the documents and the corresponding cluster centers. 
 
\paragraph{} A  paper by Jade Goldstein\cite{Publication2} discusses a text extraction approach to multi-document summarization that builds on single-document summarization methods by using additional, available information about the document set as a whole and the relationships between the documents. His approach addresses these issues by using domain independent techniques based mainly on fast, statistical processing, a metric for reducing redundancy and maximizing diversity in the selected passages, and a modular framework to allow easy parameterization for different genres, corpora characteristics and user requirements.
 
\paragraph{} Ordering information is difficult but an important task for application generating natural language texts such as multi-document summarization. In Multi-doument Summarization information is selected from a set of source documents.Therefore the optimal ordering of those selected pieces of information to create coherent summary is not obvious.To capure the preference of a sentence againts another sentence, 5 prefernce experets were intoduced by Danushka Bollegala  chronology, probablistic, topical-closeness, precedence and succession.The proposed sentence ordering algorithm considers pairwise comparison between sentences to determine a total ordering , using a greedy algorithm, thereby avoiding the combinatorial time complexity typically associated wih total ordering task. 
 
 \paragraph{} A preference learning approach to sentence ordering for multi - document summarization